%! Author = ander
%! Date = 18-10-2024
\section{Attacks on Ethereum}\label{sec:attacks-on-ethereum}
\subsection{Reorg}\label{subsec:reorg}

\subsection{DoS}\label{subsec:dos}
We've found three different kinds of DoS attacks that either were or are possible to perform on Ethereum.

One of the attacks is called \textit{under-priced opcodes}~\cite{10.1145/3391195,9815256}.
This attack works because Ethereum has a gas mechanism to reduce abuse of computing resources.
Though when a contract has a lot of underpriced opcodes, they will consume many resources.
Execution of contracts requires a lot of resources.

To mitigate this,
Ethereum has raised the gas cost of opcodes
to preserve the number of transactions-per-second~\cite{Opcode-mitigation}~\footnote{
\href{https://github.com/ethereum/EIPs/blob/master/EIPS/eip-150.md}{EIPs/EIPS/eip-150.md at master · ethereum/EIPs · GitHub}}.


Another attack,
which is closely related to the former, is \textit{empty account in the state trie}~\cite{10.1145/3391195,9815256}.
This attack was possible
because the existence of empty accounts increases the transaction processing time and synchronization.
An empty account is an account with zero balance and no code.
The attack required the proposer to select only the transactions of the adversary,
which could be insured by offering a higher gas price.

The mitigation is a combination of the one
explained for \textit{under-priced opcodes} as well as a mitigation
for clearing empty accounts~\cite{Opcode-mitigation,empty-account-mitigation,empty-account-eip-mitigation}~\footnote{
\href{https://github.com/ethereum/EIPs/blob/master/EIPS/eip-161.md}{EIPs/EIPS/eip-161.md at master · ethereum/EIPs · GitHub}}.








\subsection{Balancing Attack}\label{subsec:balancing-attack}

\subsection{Finality Attack (Bouncing Attack)}\label{subsec:finality-attack-(bouncing-attack)}

\subsection{Avalanche Attack}\label{subsec:avalanche-attack}

\subsection{Bribery}\label{subsec:bribery}

\subsection{Staircase Attack}\label{subsec:staircase-attack}