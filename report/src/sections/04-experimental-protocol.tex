

\section{Experimental protocol}\label{sec:experimental-protocol}
In this section, we describe the setup of the experiment that will be conducted as well as the experiment itself.
The section will also cover the workings of the attack that will ve performed.

\subsection{Setup}\label{subsec:setup}
The setup of the experiment is as follows: First we modified a prysm client to include logging of attestations received by the client and all the peer that the node has discovered.
The client was then run on a virtual machine hosted on Strato Claaudia through the Aalborg University.
To start receiving attestations and reaching and connecting other peers, the client was connected to the Holesky testnet.
Since the experiment is of an adversarial nature, the client was connected to a testnet insted of the mainnet.
While the client was running, it was able to log attestations received and peers discovered gaining their IPs, the subnets that the attestations were sent to, the subnet that the peer was in, the public key associated with the attestation and more.
\todo {mabye add image of tableheaders here for vizualization}
The data, that was logged, was then stored in a database for further analysis.
Using this data different peers sorted into categories depending on things like the attestation was sent to a subnet that the peer already was subscribed to or not, if the peer is subscribed to every subnet and if the peer never sent an attestation.
When it comes to deanonymizing the validators, we used the same heuristic as in the paper by~\cite{heimbach2024deanonymizingethereumvalidatorsp2p}. \todo{Paper name maybye}
The heuristic includes four criteria being:
\begin{itemize}
    \item The proportion of non-backbone attestations for validator v exceed
    \begin{equation}
                                                                                 0.9*\left(\frac{64-n_{sub}(p)}{64}\right)
                                                                                 \label{eq:heurestic}
    \end{equation} where nsub(p) is the average number of subnets the peer is subscribed over the connection’s duration.
    \item The peer is not subscribed to all 64 subnets.
    \item TWe receive at least every tenth attestation we expect for the validator v from the peer.
    \item The number of attestations we receive for the validator v from the peer p exceeds the mean number of attestations per validator from peer p by two standard deviations.
\end{itemize}
\todo{is it fine to just copy the criteria from the paper word for word?}
The first criterion is that only look at the peers which we receive the most attestations from.
The second criterion is there because if the peer is subscribed to every subnet then we will never receive a non-backbone attestation form that peer, which makes it hard to deanonymize any validates on the peer.
The third criterion is there to make sure that we have enough data form the peer.
The last criterion is there to make sure that the peer is not participating in any uncommon or non-standard behavior.
In the cases where all the criteria are met, we can be confident that the peer is hosting validators.


\subsection{Attack implementation}\label{subsec:attack-implementation}
As mentioned in x~\todo{reference to subsection above},
the work in this paper revolves around modifying a fork of the Prysm consensus client.
An important thing to keep in mind is that the core functionality of Prysm is never changed.
So no changes have been made to variables already being used in the Prysm client.

Instead, to be able to implement the attack,
it is a case of finding where the different functionalities are written in the code.
It is enough to modify just two classes in the code to reach a point where the attack is possible.

As mentioned in~\todo{reference to subsection above} two entities are logged, peers and attestations.
This translates almost directly to the code in the Prysm client, as each of these can be handled by a class each. \todo{How much are we allowed to say? We assume mentioning which classes/functions to modify is unethical}

\subsubsection{Peers}\label{subsubsec:peers}
The logs of the attestations share similarities with the logs of the peers, so the peers will be described first.

A pointer to a \texttt{Service} object is typically maintained in the different classes of the Prysm code.
This object is an implementation of an interface responsible for handling functionality coupled with the given class.
The class that handles~\gls{p2p} connections has got this as well, with several specific functions for handling a peer.

Because of said object, information about all earlier or connected peers is available.
Set to run every minute, the modified code requests all peers and iterates through them,
logging the information mentioned in~\todo{reference above subsection}.


Also in each iteration,
the collected information is sent to an \texttt{sqlite3} database such the data can be processed later.

\subsubsection{Attestations}\label{subsubsec:attestations}
Everything logged in for the peers is also logged for the attestations,
as there is always an underlying node sending each attestation.

The class handling the attestations also has a \texttt{Service} interface implementation.
This implementation primarily includes functions specific for the attestations, and not peers.
Therefore, each \texttt{Service} object has a pointer to a configuration of the running chain.
Referencing this pointer allows the class to access functions as the ones available in the~\gls{p2p} class.

Instead of being logged for once every minute, as in~\autoref{subsubsec:peers},
we log information about an attestation whenever we receive one.

With the proposer DoS attack in mind, we are interested in receiving the public key of the validator,
who is the creator of the attestation.
Getting this requires little effort,
as there already exists a function which uses the attestation's data to get public keys.
We can use this public key in another existing function to get the ID of the validator.

After collection all the information available for an attestation,
the data is logged to a database as in~\autoref{subsubsec:peers}.